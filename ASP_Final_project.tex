\documentclass[11pt]{article}
\usepackage{./reportNTU}
\usepackage{amsmath}
\usepackage{algorithm}
\usepackage{algpseudocode}
\usepackage{graphicx}

\pagestyle{fancyplain}

 
\rhead{\sffamily 1/1/2024}
\lhead{\sffamily Final}
\chead{{\sffamily ASP Fall,2023}}
\lfoot{}
\cfoot{\thepage}
\rfoot{}


\setlength\parskip{1ex}

\title{ASP Final Project}
\author{YAO-MING CHEN}
\date{Student ID:R12942136}
\begin{document}
\maketitle

\thispagestyle{fancyplain}
\renewcommand{\footrulewidth}{0.4pt}
%\thispagestyle{plain}

%Question 1
\begin{mdframed}[backgroundcolor=lightshadecolor, linewidth=0pt, innertopmargin=12pt, innerbottommargin=12pt]
{\bf Question 1.}:What are the details of these beamformers? Please describe the details using
mathematical equations and proper explanations. Also define all the notations properly.\newline
1.The beamformer with uniform weights\newline
2.The beamformer with array steering\newline
3.The MVDR beamformer\newline
4.The LCMV beamformer
\end{mdframed}
Ans.\newline
Suppose that $\textbf{x}(t)$ is our array data model, and the beamformer outcome $y(t) = \textbf{w}^H\textbf{x}(t)$\newline
1. Uniform weights mean that the weights are 1/N, where N is the number of antennas on our ULA model.\newline
Note that this beamforming is base on the assumption of DOA is 0 degree.\newline
Now, we can write the outcome as follow:
$$y(t) = (\textbf{w}^H \textbf{a}(\theta)) s_1(t) + \textbf{w}^H \textbf{n}(t)$$
Then we can derive the SNR as below:
$$SNR = |B_\theta(\theta)|^2 \frac{N\sigma_1^2}{\sigma_n^2}$$
where $|B_\theta(\theta)|^2 = \textbf{w}^H \textbf{a}(\theta)$\newline
We can clearly see that the value of SNR is highly relevant to $|B_\theta(\theta)|$ and N.\newline
2. Derived from the model of ULA, the waveforms received at elements can be expressed as:
$$\begin{bmatrix} s_0(t) \\ s_1(t) \\ \vdots \\ s_{(N-1)(t)}\end{bmatrix} = 
s_O(t) \begin{bmatrix} e^{j2\pi(\frac{d_0}{\lambda})sin\theta} \\ e^{j2\pi(\frac{d_1}{\lambda})sin\theta} \\ \vdots \\  e^{j2\pi(\frac{d_{N-1}}{\lambda})sin\theta} \end{bmatrix} = \textbf{a}(\theta) s_O(t)$$
where $\theta$ is the transmission direction, $\lambda$ is the wavelength of our carrier and $d_i, i = 1 \sim N-1$ is the distance between the first antenna and the $i_{th}$ antenna.\newline
Note that $\textbf{a}(\theta)$ is called the sheering vector.\newline
Combined with the Uniform Beamforming, we can choose the weight vector as:
$$\textbf{w} = \frac{1}{N} \textbf{a}(\theta)$$


\noindent3. MVDR(Minimum Variance Distortionless Response) Beamforming initially aim to solve the following optimization problem:

\begin{equation*}
\begin{aligned}
&\underset{w}{min} && \mathbb{E}[|\textbf{w}^H \textbf{n}(t)|^2] \\ &sub. &&\textbf{w}^H \textbf{a}(\theta_s) = 1
\end{aligned}
\end{equation*}

Since $\mathbb{E}[\textbf{n}(t)^H \textbf{n}(t)]$ is unavailable, by adding some constraints we can rewrite the problem into following format:

\begin{equation*}
\begin{aligned}
&\underset{w}{min} && \mathbb{E}[|y(t)|^2] \\ &sub. &&\textbf{w}^H \textbf{a}(\theta_s) = 1 \\ &&& y(t) = \textbf{w}^H \textbf{x}(t) \\ &&&\mathbb{E}[\textbf{n}(t) s_1^*(t)] = \textbf{0}
\end{aligned}
\end{equation*}
By solving this problem, we can write down the optimal solution:
$$\textbf{w}_{MVDR}= \frac{\textbf{R}^{-1} \textbf{a}(\theta_s)}{\textbf{a}^H(\theta_s)\textbf{R}^{-1} \textbf{a}(\theta_s)}$$
where $\textbf{R} = \mathbb{E}[\textbf{x}(t) \textbf{x}^H(t)]$\newline

\noindent 4. LCMV(Linearly Constrained Minimum Variance) beamforming is a general version of MVDR beamforming.
By changing constraint, we can write the optimization problem as follow:
\begin{equation*}
\begin{aligned}
&\underset{w}{min} && \mathbb{E}[|y(t)|^2] \\ &sub. &&\textbf{C}^H \textbf{w} = \textbf{g} \\ &&& y(t) = \textbf{w}^H \textbf{x}(t) \\ &&&\mathbb{E}[\textbf{n}(t) s_1^*(t)] = \textbf{0}
\end{aligned}
\end{equation*}
Note that when $\textbf{C} = \textbf{a}(\theta_s)$ and $\textbf{g} = \textbf{1}$, LCMV is equivalent to MVDR.
By solving this problem, we can write down the optimal solution:
$$\textbf{w}_{LCMV}= \textbf{R}^{-1} \textbf{C} (\textbf{C}^{H} \textbf{R}^{-1} \textbf{C})^{-1} \textbf{g}$$
where $\textbf{R} = \mathbb{E}[\textbf{x}(t) \textbf{x}^H(t)]$\newline

%Question 2
\begin{mdframed}[backgroundcolor=lightshadecolor, linewidth=0pt, innertopmargin=12pt, innerbottommargin=12pt]
{\bf Question 2.}:Design a filter to denoise $\Tilde{\theta}_s(t)$ and $\Tilde{\theta}_i(t)$. The denoised results are denoted by $\hat{\theta}_s(t)$ and $\hat{\theta}_i(t)$.\newline
1. Describe the details of your filter using mathematical equations.\newline
2. Summarize the steps of your filter similar to that in Page 39 of 10 LS RLS.pdf.\newline
3. What are the advantages of your filter?
\end{mdframed}
Ans.\newline
In this Item, I will use Hilber-Huang transform to filter out the noise.\newline
Hilber-Huang transform is a method that can decomposite the original signal into several sinusoid-like components and trends signal.\newline
The procedure can be seem as follow:\newline
Assume that y(t) is our noisy signal.\newline
1. Find local peaks and connect them(usually using B-spline) as p(t).\newline
2. Find local dips and connect them(usually using B-spline) as d(t).\newline
3. Compute z(t) = (p(t) + d(t))/2.\newline
4. Check if $\mathbb{E}[z(t)] \simeq 0$. If True, set y(t) as y(t) - z(t) and repeat Step1~4. If False, end procedure.\newline
Note that in realization, we can set up a threshold $\lambda$, where $\lambda \simeq 0$,to fullfill the checking process in step4.\newline
See the algorithm down below:

\begin{algorithm}
    \caption{Hilber-Huang transform}\label{alg:cap}
    \begin{algorithmic}
    \Require The noisy signal $x(t)$and the threshold $\emph{thr}$
    \State $K \gets 1000$
    \State $y(t) \gets x(t)$
    \State $z(t) \gets \textbf{0}$
    \For{i = 1,2,...,K}
        \State $z(t) = \frac{e_{max}(t) + e_{min}(t)}{2}$ \Comment $e_{max}(t),e_{min}(t)$ stands for local peaks and dips of $y(t)$
        \If{$|\mathbb{E}[z(t)]| < \emph{thr}$}
            \State $y(t) = y(t) - z(t)$
        \Else
            \State break
        \EndIf
    \EndFor
    \end{algorithmic}
\end{algorithm}
\noindent This method has the advantage of not using Fourier transform and matrix multiplication, which cost a very small amount of computations.\newline
However, this method can only apply on a noisy signal which was corrupt by sinusoid-like noised with zeros means.

%Question 3
\begin{mdframed}[backgroundcolor=lightshadecolor, linewidth=0pt, innertopmargin=12pt, innerbottommargin=12pt]
    {\bf Question 3.}:Plot the estimated DOAs $\hat{\theta}_s(t)$ and $\hat{\theta}_i(t)$ over the time index t, using your filter in Item 2.
\end{mdframed}
Ans.\newline
The filtered result in $\textbf{Figure}\,1.$\newline
\begin{figure}[h]
    \centering
    \includegraphics[width=14cm, height = 7cm]{ASP_Final_DOA.jpg} 
    \caption{$\hat{\theta}_s(t)\; and\; \hat{\theta}_i(t)$}
\end{figure}

%Question 4
\begin{mdframed}[backgroundcolor=lightshadecolor, linewidth=0pt, innertopmargin=12pt, innerbottommargin=12pt]
    {\bf Question 4.}:Design a beamformer that extracts the source signal in (1). The extracted result is denoted by $\hat{s}(t)$
\end{mdframed}
Ans.\newline
Before We start to design the beamformer, we can first take a look at the beamformers in Item 1 and discuss their pros and cons.\newline
1. $\textbf{Uniform beamforming with array steering}$: this method can enhance the SNR at certain derection and attenuate the power of noise by N factor but it can't fully filter out the interference signal(namely $i(t)$ steered by $\theta_i(t)$ in this project). See the simulation result in $Figure\;2$ \newline
2. $\textbf{MVDR}$: this method can extractd the original signal without distortion, but it can't filter out $i(t)$ and the attenuation of noise power is not guaranteed.See the simulation result in $Figure\;3$ \newline
3. $\textbf{LCMV}$: this method can be seen as a more general version of MVDR, it can strongly filter out $i(t)$, but the attenuation of noise power is not guaranteed neither. See the simulation result in $Figure\;4$ \newline
\begin{figure}[h]
    \centering
    \includegraphics[width=10cm, height = 5cm]{ASP_Final_Uni.jpg} 
    \caption{$Uniform\;Beamforming with Array Steering$}
\end{figure}

\begin{figure}[h]
    \centering
    \includegraphics[width=10cm, height = 5cm]{ASP_Final_MVDR.jpg} 
    \caption{$MVDR$}
\end{figure}
\newpage
\begin{figure}[h]
    \centering
    \includegraphics[width=10cm, height = 5cm]{ASP_Final_LCMV.jpg} 
    \caption{$LCMV$}
\end{figure}
We can also compare the three simulation results as shown in $Figure\;5$\newline
\begin{figure}[h]
    \centering
    \includegraphics[width=10cm, height = 5cm]{ASP_Final_Compare.jpg} 
    \caption{$Comparsion$}
\end{figure}
From the above results, we can clearly see that the three simulations are actually pretty similar.\newline
One of the interested things is that although the $\textbf{Uniform \;beamforming}$ can not fully filter out the interference term ($i(t)$), it's surprising that it seems like not affected by $i(t)$ very much (see the comparison result). However, the result from $\textbf{MVDR}$ and $\textbf{LCMV}$ seems to be affected by noise 
strongly as we expected.\newline

According to former discussion, I want to propose a simple method which can combined the advantages from $\textbf{Uniform beamforming}$ and $\textbf{LCMV}$ using the following procedure:\newline
1. Compute the weight vector $w_{uni}(t)$ using $\textbf{Uniform beamforming}$.\newline
2. Compute the weight vecotr $w_{LCMV}(t)$ using $\textbf{LCMV}$.\newline
3. Set a constant value $\lambda$ where $\lambda \in [0,1]$.\newline
4. Compute the weight vector $\hat{w}(t) = \lambda w_{uni}(t) + (1-\lambda)w_{LCMV}(t)$\newline
5. Compute the hypothesis true signal $\hat{s}(t) = \hat{w}(t)^H x(t)$\newline
Note that the more constant $\lambda$ close to 1, the more attenuation on noise power and the more constant $\lambda$ close to 0, the more attenuation on interference signal.\newline
The algorithm of this proposed procedure:\newpage
\begin{algorithm}
    \caption{Beamforming combined with Uniform Beamforming and LCMV}\label{alg:cap}
    \begin{algorithmic}
    \Require received signal from ULA $\textbf{x}(t)$, $\hat{\theta}_s(t)$, $\hat{\theta}_i(t)$, constant $\lambda$
    \State $R \gets \frac{1}{L} \sum_{i=1}^{L}\textbf{x}(i) \textbf{x}(i)^H$ \Comment L stands for number of measurement vector
    \State $\textbf{w}_{uni}(t) \gets \textbf{0}_{N \times L}$  \Comment N stants for number of antennas
    \State $\textbf{w}_{LCMV}(t) \gets \textbf{0}_{N \times L}$
    \State $\hat{\textbf{w}}(t) \gets \textbf{0}_{N \times L}$
    \State $\textbf{g} \gets [1, 0]^T$
    \State $\hat{s}(t) \gets \textbf{0}_{1\times L}$
    \For{k = 1,2,...,L}
        \State $\textbf{a}_s = [1, e^{j\pi\sin{\hat{\theta}_s(k)}},\dots,e^{j\pi (N-1) \sin{\hat{\theta}_s(k)}}]^T$
        \State $\textbf{a}_i = [1, e^{j\pi\sin{\hat{\theta}_i(k)}},\dots,e^{j\pi (N-1) \sin{\hat{\theta}_i(k)}}]^T$ 
        \State $\textbf{C} = [\textbf{a}_s, \textbf{a}_i]$
        \State $\textbf{w}_{uni}(i)_{N\times 1} = \frac{1}{N} \textbf{a}_s$
        \State $\textbf{w}_{LCMV}(i)_{N\times 1} = \textbf{R}^{-1} \textbf{C} (\textbf{C}^H \textbf{R}^{-1} \textbf{C})^{-1} \textbf{g}$
        \State $\hat{\textbf{w}}(i)_{N\times 1} = \lambda \textbf{w}_{uni}(i) + (1 - \lambda) \textbf{w}_{LCMV}(i)$
        \State $\hat{s}(i) = \hat{\textbf{w}}(i)^H \textbf{x}(i)_{N \times 1}$
    \EndFor
    \end{algorithmic}
\end{algorithm}
In the following simulation, I will set $\lambda = 0.8$ since the interference signal are relevantly less affected compared to noise in this project.\newline

\begin{mdframed}[backgroundcolor=lightshadecolor, linewidth=0pt, innertopmargin=12pt, innerbottommargin=12pt]
    {\bf Question 5.}:Plot the $\emph{real and imaginary parts}$ of estimated source signal $\hat{s}(t)$ over the time index t, using your beamformer in Item 4.
\end{mdframed}
Ans.\newline
Result is shown in $Figure\;6$\newline
\begin{figure}[h]
    \centering
    \includegraphics[width=16cm, height = 8cm]{ASP_Final_hat.jpg} 
    \caption{$\hat{s}(t)$}
\end{figure}
%\bibliographystyle{IEEEtran}
%\bibliography{./Ref.bib}

\end{document}